% Options for packages loaded elsewhere
\PassOptionsToPackage{unicode}{hyperref}
\PassOptionsToPackage{hyphens}{url}
%
\documentclass[
]{article}
\usepackage{amsmath,amssymb}
\usepackage{iftex}
\ifPDFTeX
  \usepackage[T1]{fontenc}
  \usepackage[utf8]{inputenc}
  \usepackage{textcomp} % provide euro and other symbols
\else % if luatex or xetex
  \usepackage{unicode-math} % this also loads fontspec
  \defaultfontfeatures{Scale=MatchLowercase}
  \defaultfontfeatures[\rmfamily]{Ligatures=TeX,Scale=1}
\fi
\usepackage{lmodern}
\ifPDFTeX\else
  % xetex/luatex font selection
\fi
% Use upquote if available, for straight quotes in verbatim environments
\IfFileExists{upquote.sty}{\usepackage{upquote}}{}
\IfFileExists{microtype.sty}{% use microtype if available
  \usepackage[]{microtype}
  \UseMicrotypeSet[protrusion]{basicmath} % disable protrusion for tt fonts
}{}
\makeatletter
\@ifundefined{KOMAClassName}{% if non-KOMA class
  \IfFileExists{parskip.sty}{%
    \usepackage{parskip}
  }{% else
    \setlength{\parindent}{0pt}
    \setlength{\parskip}{6pt plus 2pt minus 1pt}}
}{% if KOMA class
  \KOMAoptions{parskip=half}}
\makeatother
\usepackage{xcolor}
\usepackage[margin=1in]{geometry}
\usepackage{color}
\usepackage{fancyvrb}
\newcommand{\VerbBar}{|}
\newcommand{\VERB}{\Verb[commandchars=\\\{\}]}
\DefineVerbatimEnvironment{Highlighting}{Verbatim}{commandchars=\\\{\}}
% Add ',fontsize=\small' for more characters per line
\usepackage{framed}
\definecolor{shadecolor}{RGB}{248,248,248}
\newenvironment{Shaded}{\begin{snugshade}}{\end{snugshade}}
\newcommand{\AlertTok}[1]{\textcolor[rgb]{0.94,0.16,0.16}{#1}}
\newcommand{\AnnotationTok}[1]{\textcolor[rgb]{0.56,0.35,0.01}{\textbf{\textit{#1}}}}
\newcommand{\AttributeTok}[1]{\textcolor[rgb]{0.13,0.29,0.53}{#1}}
\newcommand{\BaseNTok}[1]{\textcolor[rgb]{0.00,0.00,0.81}{#1}}
\newcommand{\BuiltInTok}[1]{#1}
\newcommand{\CharTok}[1]{\textcolor[rgb]{0.31,0.60,0.02}{#1}}
\newcommand{\CommentTok}[1]{\textcolor[rgb]{0.56,0.35,0.01}{\textit{#1}}}
\newcommand{\CommentVarTok}[1]{\textcolor[rgb]{0.56,0.35,0.01}{\textbf{\textit{#1}}}}
\newcommand{\ConstantTok}[1]{\textcolor[rgb]{0.56,0.35,0.01}{#1}}
\newcommand{\ControlFlowTok}[1]{\textcolor[rgb]{0.13,0.29,0.53}{\textbf{#1}}}
\newcommand{\DataTypeTok}[1]{\textcolor[rgb]{0.13,0.29,0.53}{#1}}
\newcommand{\DecValTok}[1]{\textcolor[rgb]{0.00,0.00,0.81}{#1}}
\newcommand{\DocumentationTok}[1]{\textcolor[rgb]{0.56,0.35,0.01}{\textbf{\textit{#1}}}}
\newcommand{\ErrorTok}[1]{\textcolor[rgb]{0.64,0.00,0.00}{\textbf{#1}}}
\newcommand{\ExtensionTok}[1]{#1}
\newcommand{\FloatTok}[1]{\textcolor[rgb]{0.00,0.00,0.81}{#1}}
\newcommand{\FunctionTok}[1]{\textcolor[rgb]{0.13,0.29,0.53}{\textbf{#1}}}
\newcommand{\ImportTok}[1]{#1}
\newcommand{\InformationTok}[1]{\textcolor[rgb]{0.56,0.35,0.01}{\textbf{\textit{#1}}}}
\newcommand{\KeywordTok}[1]{\textcolor[rgb]{0.13,0.29,0.53}{\textbf{#1}}}
\newcommand{\NormalTok}[1]{#1}
\newcommand{\OperatorTok}[1]{\textcolor[rgb]{0.81,0.36,0.00}{\textbf{#1}}}
\newcommand{\OtherTok}[1]{\textcolor[rgb]{0.56,0.35,0.01}{#1}}
\newcommand{\PreprocessorTok}[1]{\textcolor[rgb]{0.56,0.35,0.01}{\textit{#1}}}
\newcommand{\RegionMarkerTok}[1]{#1}
\newcommand{\SpecialCharTok}[1]{\textcolor[rgb]{0.81,0.36,0.00}{\textbf{#1}}}
\newcommand{\SpecialStringTok}[1]{\textcolor[rgb]{0.31,0.60,0.02}{#1}}
\newcommand{\StringTok}[1]{\textcolor[rgb]{0.31,0.60,0.02}{#1}}
\newcommand{\VariableTok}[1]{\textcolor[rgb]{0.00,0.00,0.00}{#1}}
\newcommand{\VerbatimStringTok}[1]{\textcolor[rgb]{0.31,0.60,0.02}{#1}}
\newcommand{\WarningTok}[1]{\textcolor[rgb]{0.56,0.35,0.01}{\textbf{\textit{#1}}}}
\usepackage{graphicx}
\makeatletter
\def\maxwidth{\ifdim\Gin@nat@width>\linewidth\linewidth\else\Gin@nat@width\fi}
\def\maxheight{\ifdim\Gin@nat@height>\textheight\textheight\else\Gin@nat@height\fi}
\makeatother
% Scale images if necessary, so that they will not overflow the page
% margins by default, and it is still possible to overwrite the defaults
% using explicit options in \includegraphics[width, height, ...]{}
\setkeys{Gin}{width=\maxwidth,height=\maxheight,keepaspectratio}
% Set default figure placement to htbp
\makeatletter
\def\fps@figure{htbp}
\makeatother
\setlength{\emergencystretch}{3em} % prevent overfull lines
\providecommand{\tightlist}{%
  \setlength{\itemsep}{0pt}\setlength{\parskip}{0pt}}
\setcounter{secnumdepth}{-\maxdimen} % remove section numbering
\ifLuaTeX
  \usepackage{selnolig}  % disable illegal ligatures
\fi
\usepackage{bookmark}
\IfFileExists{xurl.sty}{\usepackage{xurl}}{} % add URL line breaks if available
\urlstyle{same}
\hypersetup{
  pdftitle={Wrangling and Visualizing Data with R},
  pdfauthor={Abu Ali},
  hidelinks,
  pdfcreator={LaTeX via pandoc}}

\title{Wrangling and Visualizing Data with R}
\author{Abu Ali}
\date{2024-08-08}

\begin{document}
\maketitle

\subsection{The Titanic Dataset}\label{the-titanic-dataset}

we will use R packages to explore the Titanic dataset and visualize key
patterns and insights, and their relations to the survival rate of the
passengers.

You can download the the titanic dataset here:
\url{https://www.kaggle.com/datasets/yasserh/titanic-dataset}

\subsubsection{LOADING THE DATASET:}\label{loading-the-dataset}

\begin{Shaded}
\begin{Highlighting}[]
\FunctionTok{library}\NormalTok{(readxl)}
\NormalTok{titanic\_ds }\OtherTok{\textless{}{-}} \FunctionTok{read\_excel}\NormalTok{(}\StringTok{"titanic\_ds.xls"}\NormalTok{)}
\end{Highlighting}
\end{Shaded}

\begin{verbatim}
## Warning: Coercing text to numeric in M1306 / R1306C13: '328'
\end{verbatim}

\begin{Shaded}
\begin{Highlighting}[]
\FunctionTok{str}\NormalTok{(titanic\_ds)}
\end{Highlighting}
\end{Shaded}

\begin{verbatim}
## tibble [1,309 x 14] (S3: tbl_df/tbl/data.frame)
##  $ pclass   : num [1:1309] 1 1 1 1 1 1 1 1 1 1 ...
##  $ survived : num [1:1309] 1 1 0 0 0 1 1 0 1 0 ...
##  $ name     : chr [1:1309] "Allen, Miss. Elisabeth Walton" "Allison, Master. Hudson Trevor" "Allison, Miss. Helen Loraine" "Allison, Mr. Hudson Joshua Creighton" ...
##  $ sex      : chr [1:1309] "female" "male" "female" "male" ...
##  $ age      : num [1:1309] 29 0.917 2 30 25 ...
##  $ sibsp    : num [1:1309] 0 1 1 1 1 0 1 0 2 0 ...
##  $ parch    : num [1:1309] 0 2 2 2 2 0 0 0 0 0 ...
##  $ ticket   : chr [1:1309] "24160" "113781" "113781" "113781" ...
##  $ fare     : num [1:1309] 211 152 152 152 152 ...
##  $ cabin    : chr [1:1309] "B5" "C22 C26" "C22 C26" "C22 C26" ...
##  $ embarked : chr [1:1309] "S" "S" "S" "S" ...
##  $ boat     : chr [1:1309] "2" "11" NA NA ...
##  $ body     : num [1:1309] NA NA NA 135 NA NA NA NA NA 22 ...
##  $ home.dest: chr [1:1309] "St Louis, MO" "Montreal, PQ / Chesterville, ON" "Montreal, PQ / Chesterville, ON" "Montreal, PQ / Chesterville, ON" ...
\end{verbatim}

\subsubsection{CLEANING THE DATASET:}\label{cleaning-the-dataset}

\paragraph{1. Check for missing values}\label{check-for-missing-values}

\begin{Shaded}
\begin{Highlighting}[]
\FunctionTok{library}\NormalTok{(Amelia)}
\end{Highlighting}
\end{Shaded}

\begin{verbatim}
## Loading required package: Rcpp
\end{verbatim}

\begin{verbatim}
## ## 
## ## Amelia II: Multiple Imputation
## ## (Version 1.8.2, built: 2024-04-10)
## ## Copyright (C) 2005-2024 James Honaker, Gary King and Matthew Blackwell
## ## Refer to http://gking.harvard.edu/amelia/ for more information
## ##
\end{verbatim}

\begin{Shaded}
\begin{Highlighting}[]
\FunctionTok{missmap}\NormalTok{(titanic\_ds, }\AttributeTok{col =} \FunctionTok{c}\NormalTok{(}\StringTok{"red"}\NormalTok{, }\StringTok{"green"}\NormalTok{))}
\end{Highlighting}
\end{Shaded}

\begin{verbatim}
## Warning: Unknown or uninitialised column: `arguments`.
## Unknown or uninitialised column: `arguments`.
\end{verbatim}

\begin{verbatim}
## Warning: Unknown or uninitialised column: `imputations`.
\end{verbatim}

\includegraphics{Data-Wrangling-and-Visualization-with-R_files/figure-latex/missing values-1.pdf}

Note that you can add \texttt{echo\ =\ FALSE} parameter to the code
chunk to prevent printing of the R code that generated the plot.

\paragraph{2. Select relevant columns for the
analysis}\label{select-relevant-columns-for-the-analysis}

\begin{Shaded}
\begin{Highlighting}[]
\FunctionTok{library}\NormalTok{(tidyverse)}
\end{Highlighting}
\end{Shaded}

\begin{verbatim}
## -- Attaching core tidyverse packages ------------------------ tidyverse 2.0.0 --
## v dplyr     1.1.4     v readr     2.1.5
## v forcats   1.0.0     v stringr   1.5.1
## v ggplot2   3.5.1     v tibble    3.2.1
## v lubridate 1.9.3     v tidyr     1.3.1
## v purrr     1.0.2     
## -- Conflicts ------------------------------------------ tidyverse_conflicts() --
## x dplyr::filter() masks stats::filter()
## x dplyr::lag()    masks stats::lag()
## i Use the conflicted package (<http://conflicted.r-lib.org/>) to force all conflicts to become errors
\end{verbatim}

\begin{Shaded}
\begin{Highlighting}[]
\NormalTok{selected\_titanic }\OtherTok{\textless{}{-}}\NormalTok{ titanic\_ds }\SpecialCharTok{\%\textgreater{}\%}
  \FunctionTok{select}\NormalTok{ (age, pclass, sex, survived, embarked, home.dest, fare, parch, sibsp)}
\end{Highlighting}
\end{Shaded}

\paragraph{3. Merge columns parch and sibsp to create a new column,
FamilySize}\label{merge-columns-parch-and-sibsp-to-create-a-new-column-familysize}

\begin{Shaded}
\begin{Highlighting}[]
\NormalTok{selected\_titanic}\SpecialCharTok{$}\NormalTok{FamilySize }\OtherTok{\textless{}{-}}\NormalTok{ selected\_titanic}\SpecialCharTok{$}\NormalTok{sibsp }\SpecialCharTok{+}\NormalTok{ selected\_titanic}\SpecialCharTok{$}\NormalTok{parch }\SpecialCharTok{+} \DecValTok{1}
\FunctionTok{str}\NormalTok{(selected\_titanic)}
\end{Highlighting}
\end{Shaded}

\begin{verbatim}
## tibble [1,309 x 10] (S3: tbl_df/tbl/data.frame)
##  $ age       : num [1:1309] 29 0.917 2 30 25 ...
##  $ pclass    : num [1:1309] 1 1 1 1 1 1 1 1 1 1 ...
##  $ sex       : chr [1:1309] "female" "male" "female" "male" ...
##  $ survived  : num [1:1309] 1 1 0 0 0 1 1 0 1 0 ...
##  $ embarked  : chr [1:1309] "S" "S" "S" "S" ...
##  $ home.dest : chr [1:1309] "St Louis, MO" "Montreal, PQ / Chesterville, ON" "Montreal, PQ / Chesterville, ON" "Montreal, PQ / Chesterville, ON" ...
##  $ fare      : num [1:1309] 211 152 152 152 152 ...
##  $ parch     : num [1:1309] 0 2 2 2 2 0 0 0 0 0 ...
##  $ sibsp     : num [1:1309] 0 1 1 1 1 0 1 0 2 0 ...
##  $ FamilySize: num [1:1309] 1 4 4 4 4 1 2 1 3 1 ...
\end{verbatim}

\paragraph{4. Categorize the fare column and assign label to each
category}\label{categorize-the-fare-column-and-assign-label-to-each-category}

\begin{Shaded}
\begin{Highlighting}[]
\NormalTok{selected\_titanic}\SpecialCharTok{$}\NormalTok{FareCategory }\OtherTok{\textless{}{-}} \FunctionTok{cut}\NormalTok{(selected\_titanic}\SpecialCharTok{$}\NormalTok{fare, }
                                   \AttributeTok{breaks =} \FunctionTok{c}\NormalTok{(}\DecValTok{0}\NormalTok{, }\DecValTok{10}\NormalTok{, }\DecValTok{20}\NormalTok{, }\DecValTok{50}\NormalTok{, }\DecValTok{100}\NormalTok{, }\ConstantTok{Inf}\NormalTok{), }
                                   \AttributeTok{labels =} \FunctionTok{c}\NormalTok{(}\StringTok{"Lowest"}\NormalTok{, }\StringTok{"Lower Middle"}\NormalTok{, }
                                        \StringTok{"Upper Middle"}\NormalTok{, }\StringTok{"Higher"}\NormalTok{, }\StringTok{"Highest"}\NormalTok{))}
\FunctionTok{str}\NormalTok{(selected\_titanic)}
\end{Highlighting}
\end{Shaded}

\begin{verbatim}
## tibble [1,309 x 11] (S3: tbl_df/tbl/data.frame)
##  $ age         : num [1:1309] 29 0.917 2 30 25 ...
##  $ pclass      : num [1:1309] 1 1 1 1 1 1 1 1 1 1 ...
##  $ sex         : chr [1:1309] "female" "male" "female" "male" ...
##  $ survived    : num [1:1309] 1 1 0 0 0 1 1 0 1 0 ...
##  $ embarked    : chr [1:1309] "S" "S" "S" "S" ...
##  $ home.dest   : chr [1:1309] "St Louis, MO" "Montreal, PQ / Chesterville, ON" "Montreal, PQ / Chesterville, ON" "Montreal, PQ / Chesterville, ON" ...
##  $ fare        : num [1:1309] 211 152 152 152 152 ...
##  $ parch       : num [1:1309] 0 2 2 2 2 0 0 0 0 0 ...
##  $ sibsp       : num [1:1309] 0 1 1 1 1 0 1 0 2 0 ...
##  $ FamilySize  : num [1:1309] 1 4 4 4 4 1 2 1 3 1 ...
##  $ FareCategory: Factor w/ 5 levels "Lowest","Lower Middle",..: 5 5 5 5 5 3 4 NA 4 3 ...
\end{verbatim}

\paragraph{5. Remove the columns that are being merged to form new
columns}\label{remove-the-columns-that-are-being-merged-to-form-new-columns}

\begin{Shaded}
\begin{Highlighting}[]
\NormalTok{selected\_titanic }\OtherTok{\textless{}{-}}\NormalTok{ selected\_titanic }\SpecialCharTok{\%\textgreater{}\%} 
  \FunctionTok{select}\NormalTok{(}\SpecialCharTok{{-}}\NormalTok{fare, }\SpecialCharTok{{-}}\NormalTok{parch, }\SpecialCharTok{{-}}\NormalTok{sibsp)}
\end{Highlighting}
\end{Shaded}

\paragraph{6. Change the values of columns pclass, survived, and
embarked}\label{change-the-values-of-columns-pclass-survived-and-embarked}

\begin{Shaded}
\begin{Highlighting}[]
\NormalTok{selected\_titanic }\OtherTok{\textless{}{-}}\NormalTok{ selected\_titanic }\SpecialCharTok{\%\textgreater{}\%}
  \FunctionTok{mutate}\NormalTok{(}
    \AttributeTok{survived =} \FunctionTok{ifelse}\NormalTok{(survived }\SpecialCharTok{==} \DecValTok{0}\NormalTok{, }\StringTok{"No"}\NormalTok{, }\StringTok{"Yes"}\NormalTok{),}
    \AttributeTok{age =} \FunctionTok{ifelse}\NormalTok{(age }\SpecialCharTok{\textgreater{}=} \DecValTok{18}\NormalTok{, }\StringTok{"Adult"}\NormalTok{, }\StringTok{"Child"}\NormalTok{),}
    \AttributeTok{pclass =} \FunctionTok{case\_when}\NormalTok{(}
\NormalTok{      pclass }\SpecialCharTok{==} \DecValTok{1} \SpecialCharTok{\textasciitilde{}} \StringTok{"1st"}\NormalTok{,}
\NormalTok{      pclass }\SpecialCharTok{==} \DecValTok{2} \SpecialCharTok{\textasciitilde{}} \StringTok{"2nd"}\NormalTok{,}
\NormalTok{      pclass }\SpecialCharTok{==} \DecValTok{3} \SpecialCharTok{\textasciitilde{}} \StringTok{"3rd"}
\NormalTok{    ),}
    
    \AttributeTok{embarked =} \FunctionTok{case\_when}\NormalTok{(}
\NormalTok{      embarked }\SpecialCharTok{==} \StringTok{"C"} \SpecialCharTok{\textasciitilde{}} \StringTok{"Cherbourg"}\NormalTok{,}
\NormalTok{      embarked }\SpecialCharTok{==} \StringTok{"Q"} \SpecialCharTok{\textasciitilde{}} \StringTok{"Queenstown"}\NormalTok{,}
\NormalTok{      embarked }\SpecialCharTok{==} \StringTok{"S"} \SpecialCharTok{\textasciitilde{}} \StringTok{"Southampton"}
\NormalTok{    )}
\NormalTok{  )}
\end{Highlighting}
\end{Shaded}

\paragraph{7. Change the name of column pclass to Class, and home.dest
to
Destination}\label{change-the-name-of-column-pclass-to-class-and-home.dest-to-destination}

\begin{Shaded}
\begin{Highlighting}[]
\NormalTok{selected\_titanic }\OtherTok{\textless{}{-}}\NormalTok{ selected\_titanic }\SpecialCharTok{\%\textgreater{}\%} 
  \FunctionTok{rename}\NormalTok{(}
    \AttributeTok{Class =}\NormalTok{ pclass,}
    \AttributeTok{Destination =}\NormalTok{ home.dest}
\NormalTok{  )}
\end{Highlighting}
\end{Shaded}

\paragraph{8. Capitalize the initials of all the columns
name}\label{capitalize-the-initials-of-all-the-columns-name}

\begin{Shaded}
\begin{Highlighting}[]
\NormalTok{selected\_titanic }\OtherTok{\textless{}{-}}\NormalTok{ selected\_titanic }\SpecialCharTok{\%\textgreater{}\%} 
  \FunctionTok{rename\_all}\NormalTok{(}\SpecialCharTok{\textasciitilde{}}\FunctionTok{str\_to\_title}\NormalTok{(.))}
\end{Highlighting}
\end{Shaded}

\paragraph{9. Check for missing values
again}\label{check-for-missing-values-again}

\begin{Shaded}
\begin{Highlighting}[]
\FunctionTok{missmap}\NormalTok{(selected\_titanic, }\AttributeTok{col =} \FunctionTok{c}\NormalTok{(}\StringTok{"red"}\NormalTok{, }\StringTok{"green"}\NormalTok{))}
\end{Highlighting}
\end{Shaded}

\begin{verbatim}
## Warning: Unknown or uninitialised column: `arguments`.
## Unknown or uninitialised column: `arguments`.
\end{verbatim}

\begin{verbatim}
## Warning: Unknown or uninitialised column: `imputations`.
\end{verbatim}

\includegraphics{Data-Wrangling-and-Visualization-with-R_files/figure-latex/check missing values-1.pdf}

\paragraph{10. Drop all the missing values from the
dataset}\label{drop-all-the-missing-values-from-the-dataset}

\begin{Shaded}
\begin{Highlighting}[]
\NormalTok{selected\_titanic }\OtherTok{\textless{}{-}} \FunctionTok{drop\_na}\NormalTok{(selected\_titanic)}
\end{Highlighting}
\end{Shaded}

\begin{verbatim}
## Warning: Unknown or uninitialised column: `arguments`.
## Unknown or uninitialised column: `arguments`.
\end{verbatim}

\begin{verbatim}
## Warning: Unknown or uninitialised column: `imputations`.
\end{verbatim}

\includegraphics{Data-Wrangling-and-Visualization-with-R_files/figure-latex/cleaned missingnessmap-1.pdf}

\subsubsection{EXPLORING THE DATASET:}\label{exploring-the-dataset}

\end{document}
